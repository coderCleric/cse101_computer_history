%%%%%%%%%%%%%%%%%%%%%%%%%%%%%%%%%%%%%%%%%%%%%%%%%%%%%%%%%%%%%%%%%%%%%%%%%%%%%%%%

% IEEEconf.cls file must exist in the same directory as the TeX file you want to compile
\documentclass[letterpaper, 10 pt, conference]{IEEEconf}

\title{\LARGE \bf
COMPUTER HISTORY\\
\large Group Topic Expressed In A Few Words
}

\author{Group 13\\
\small Justin Markwell\\
\small Jay Choate\\
\small Owen Sortwell\\
\small Wesley Camphouse\\
}

% Image/graphics support
\usepackage{graphicx}
\graphicspath{ {./images/} }

% Formatting for lists
\usepackage{enumitem}

% Formatting for media
\usepackage{float}
\restylefloat{table}
\restylefloat{figure}

\begin{document}


\maketitle
\thispagestyle{empty}
\pagestyle{empty}


%%%%%%%%%%%%%%%%%%%%%%%%%%%%%%%%%%%%%%%%%%%%%%%%%%%%%%%%%%%%%%%%%%%%%%%%%%%%%%%%
\section{Introduction}

%%%%%%%%%%%%%%%%%%%%%%%%%%%%%%%%%%%%%%%%%%%%%%%%%%%%%%%%%%%%%%%%%%%%%%%%%%%%%%%%
\section{CDC 6600 - 1964}
\subsection{Time Period}
The CDC 6600 has been argued to be the earliest supercomputer ever made. The CDC 6600 was produced in 1962 by Seymour Cray in a laboratory made specially so that he would not have distractions while completing the design for the machine. It was released to be sold in 1964 for the price of seven-million dollars, a hefty price tag by today's standards, and nothing short of gargantuan in 1964. Nonetheless, the National Center for Atmospheric Research purchased one of the machines the same year it was released, with it finally being delivered to them in late 1965. The 6600 was also a pioneer in another way, in that it sold copies of this machine to consumers beyond the regular military and government customers of super computer companies. The 6600's reign was ended later, however, when rival computer development company IBM (who's computer the CDC beat with the 6600 model) promised an even faster model, taking the spotlight off of CDC's computer. This computer, however, was never developed, and the CDC 6600 was later surpassed by the CDC 7600, another Seymour Cray design.
\subsection{Hardware}
\subsection{Software}

%%%%%%%%%%%%%%%%%%%%%%%%%%%%%%%%%%%%%%%%%%%%%%%%%%%%%%%%%%%%%%%%%%%%%%%%%%%%%%%%
\section{Intel Paragon - 1993}
\subsection{Time Period}
The Paragon was made by Intel at Caltech in the 90's, with the computer being launched in 1992. According to some, this machine was "the first massively parallel processor supercomputer to be indisputably the fastest system in the world" (Staff, 2019). This computer was run on 2048 Intel i860 RISC microprocessors, with the original intention being that each processor would help run a specialized operating system meant for this computer. However, this strategy did not work very well, and a kernel was developed by Sandia National Labs to be used with the machine. Despite the Paragon's initial success, Intel stopped development of them shortly after their release, due to the technical and financial difficulties of developing parallel computers.
\subsection{Hardware}
\subsection{Software}

%%%%%%%%%%%%%%%%%%%%%%%%%%%%%%%%%%%%%%%%%%%%%%%%%%%%%%%%%%%%%%%%%%%%%%%%%%%%%%%%

\section{Summit (IBM) - 2019}
\subsection{Time Period}
The current most powerful supercomputer in today's world is IBM's Summit, as ranked by the Department of Energy's Oak Ridge National Laboratory (ORNL). Summit is capable of performing a gargantuan number of calculations, able to do "more than three billion billion mixed precision calculations per second, or 3.3 exaops" (McCorkle, 2018). This allows this computer to aid in many different topics of research, from energy to artificial intelligence. Some other areas Summit is used in include astrophysics, materials, cancer surveillance, and system biology. With it's enormous power compared to other computers, it has been and will continue to provide new insights into whatever fields of research it is used in.
\subsection{Hardware}
\subsection{Software}

%%%%%%%%%%%%%%%%%%%%%%%%%%%%%%%%%%%%%%%%%%%%%%%%%%%%%%%%%%%%%%%%%%%%%%%%%%%%%%%%

\section{CONCLUSION}

%%%%%%%%%%%%%%%%%%%%%%%%%%%%%%%%%%%%%%%%%%%%%%%%%%%%%%%%%%%%%%%%%%%%%%%%%%%%%%%%

\section*{REFERENCES}

\begin{enumerate}[label={[\arabic*]}]
\item Supercomputers. (n.d.). In computerhistory.org. Retrieved October 6, 2019, from https://www.computerhistory.org/revolution/supercomputers/10/76
\item CDC 6600. (n.d.). In cisl.ucar.edu. Retrieved October 6, 2019, from https://www.computerhistory.org/revolution/supercomputers/10/76
\item Staff. (2019, April 17). Vintage Video: The Paragon Supercomputer – A Product of Partnership. In insidehp.com. Retrieved October 6, 2019, from https://insidehpc.com/2019/04/vintage-video-the-paragon-supercomputer-a-product-of-partnership/
\item McCorkle, M. L. (2018, June 8). ORNL lauches Summit supercomputer. In ornl.gov. Retrieved October 6, 2019, from https://www.ornl.gov/news/ornl-launches-summit-supercomputer
\end{enumerate}

\end{document}

