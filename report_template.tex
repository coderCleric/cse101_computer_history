%%%%%%%%%%%%%%%%%%%%%%%%%%%%%%%%%%%%%%%%%%%%%%%%%%%%%%%%%%%%%%%%%%%%%%%%%%%%%%%%
\documentclass[letterpaper, 10 pt, conference]{IEEEconf}

\title{\LARGE \bf
COMPUTER HISTORY\\
\large Supercomputers
}

\author{Group 13\\
\small Justin Markwell (Spikeums)\\
\small Jay Choate (BlueJay)\\
\small Owen Sortwell (PigeonStudent)\\
\small Wesley Camphouse (CoderCleric)\\
}

% Image/graphics support
\usepackage{graphicx}
\graphicspath{ {./images/} }

% Formatting for lists
\usepackage{enumitem}

% Formatting for media
\usepackage{float}
\restylefloat{table}
\restylefloat{figure}

\begin{document}


\maketitle
\thispagestyle{empty}
\pagestyle{empty}


%%%%%%%%%%%%%%%%%%%%%%%%%%%%%%%%%%%%%%%%%%%%%%%%%%%%%%%%%%%%%%%%%%%%%%%%%%%%%%%%

\section{Introduction}
Our group chose Supercomputers as our topic. We chose this simply because we thought it was interesting and we wanted to know more about them. A supercomputer is a computer that is much faster and more powerful than the average computers of the time period. They were usually used to do research and to run calculations that people could not do easily themselves. Often times supercomputers functioned as nodes, or groups, of computers. Another difference between normal and supercomputers is that supercomputers measure their computing power in flops such as gigaflops, petaflops, and so on rather than bytes like megabytes and gigabytes. Supercomputers also take in a massive amount of power. We decided to go with an approach to the project where we divide the topics under each of the three supercomputers we chose. We chose the CDC 6600 from 1964, the Intel Paragon from 1993, and the IBM Summit from 2019. 

%%%%%%%%%%%%%%%%%%%%%%%%%%%%%%%%%%%%%%%%%%%%%%%%%%%%%%%%%%%%%%%%%%%%%%%%%%%%%%%%

\section{CDC 6600 - 1964}
\subsection{Time Period}
The CDC 6600 has been argued to be the earliest supercomputer ever made. The CDC 6600 was produced in 1962 by Seymour Cray in a laboratory made specially so that he would not have distractions while completing the design for the machine. It was released to be sold in 1964 for the price of seven-million dollars, a hefty price tag by today's standards, and nothing short of gargantuan in 1964. Nonetheless, the National Center for Atmospheric Research purchased one of the machines the same year it was released, with it finally being delivered to them in late 1965. The 6600 was also a pioneer in another way, in that it sold copies of this machine to consumers beyond the regular military and government customers of super computer companies. The 6600's reign was ended later, however, when rival computer development company IBM (who's computer the CDC beat with the 6600 model) promised an even faster model, taking the spotlight off of CDC's computer. This computer, however, was never developed, and the CDC 6600 was later surpassed by the CDC 7600, another Seymour Cray design.
\subsection{Hardware}
The CDC 6600 was widely considered to be one of the very first super computers. By today's standards its hardware is far from even common laptops. It only had 1 CPU with 1 core running at 40MHz. Today its common to see computers running with 4-8 cores all running at around 4GHz, a several orders of magnitude of improvement from the CDC 6600. Interestingly this computer produced so much heat that it had to be cooled with Freon cooling system that is very similar to water cooling systems used in enthusiast and super computers alike today. All of these systems combined to produce a system that could output just over 3 million flops (floating point operations) per second.   
\subsection{Software}
NCAR's Computing Facility first acquired the CDC 6600 in 1964. 
Only a year later in 1965, the computing facility was behind in 
development for software for the CDC 6600. The two operating 
system available for use on the CDC were SPIROS (Simultaneous
Processing Operating System) and COS (Chippewa Operating System). 
Both operating systems were inefficient at running the code 
developed by NCAR. Therefore, they determined the computer would 
need it's own OS, and one was built using a 6600 simulator for 
another computer, the CDC 3600. In the end, before it's 
decommission, the CDC 6600 had it's own OS, program library, and 
FORTRAN 66 compiler. It also ran some early video games made by 
the NCAR engineers, including games similar to Lunar Lander, Space
Wars, and baseball.

%%%%%%%%%%%%%%%%%%%%%%%%%%%%%%%%%%%%%%%%%%%%%%%%%%%%%%%%%%%%%%%%%%%%%%%%%%%%%%%%

\section{Intel Paragon - 1993}
\subsection{Time Period}
The Paragon was made by Intel at Caltech in the 90's, with the computer being launched in 1992. According to some, this machine was "the first massively parallel processor supercomputer to be indisputably the fastest system in the world" (Staff, 2019). This computer was run on 2048 Intel i860 RISC microprocessors, with the original intention being that each processor would help run a specialized operating system meant for this computer. However, this strategy did not work very well, and a kernel was developed by Sandia National Labs to be used with the machine. Despite the Paragon's initial success, Intel stopped development of them shortly after their release, due to the technical and financial difficulties of developing parallel computers.
\subsection{Hardware}
The Intel Paragon was a clear step towards modern super computers used today. Rather than using a single CPU like the CDC 6600, the Paragon used up to 4000 CPUs in parallel to improve computing efficiency and total power of the system. The Paragon was one of the early adopters of the node based super computer architecture that is still widely used in the most powerful computers of today. Instead of functioning as a single computer, the super computer functions as clusters (nodes) of computers in order to better divide up the power of the entire network more effectively. The Paragon was recorded as being able to compute at 143.40 gigaflops per second. 
\subsection{Software}
The Intel Paragon was supplied by Intel to Sandia National Labs in
1994, with an OS designed by Intel. The OS provided by Intel,
OSF-1, failed to work well with the scale of the project at Sandia
National Labs. Instead of using OSF-1, the engineers at Sandia
National Labs ported their own OS over to the Intel Paragon,
SUNMOS. The kernel was designed specifically to be lightweight,
and it was later replaced by another lightweight kernel, PUMA.
Lightweight kernels were used, presumably, to save processing
power, so that the power of the supercomputer could be maximized.

%%%%%%%%%%%%%%%%%%%%%%%%%%%%%%%%%%%%%%%%%%%%%%%%%%%%%%%%%%%%%%%%%%%%%%%%%%%%%%%%

\section{Summit (IBM) - 2019}
\subsection{Time Period}
The current most powerful supercomputer in today's world is IBM's Summit, as ranked by the Department of Energy's Oak Ridge National Laboratory (ORNL). Summit is capable of performing a gargantuan number of calculations, able to do "more than three billion billion mixed precision calculations per second, or 3.3 exaops" (McCorkle, 2018). This allows this computer to aid in many different topics of research, from energy to artificial intelligence. Some other areas Summit is used in include astrophysics, materials, cancer surveillance, and system biology. With it's enormous power compared to other computers, it has been and will continue to provide new insights into whatever fields of research it is used in.
\subsection{Hardware}
The Summit is the most powerful super computer in the world at the time of writing. While using all of its 27,648 NVidia Volta V100 GPUs at once it can produce over 200 petaflops per second. The system consists of 4,608 compute nodes that each individually can compute over 42 terraflops per second. The system is cooled by a complex water cooling loop to increase the density of the supercomputer and increase the amount of computing power available per server rack. When operating at full power the entire computer network draws over 13 megawatts of power.
\subsection{Software}
For security reasons, Oak Ridge National Laboratories newest
supercomputer runs on an unknown and unreleased OS. The specific
software being used on the Summit is also unknown, however the
areas being affected by the computer are know. The computer's main
purpose is to run complex modeling programs in various scientific
fields, including biology, materials, and of course, computer
science. The Summit is the first computer to complete an exascale
(billion billion calculations per second) scientific calculation
in genomics. The Summit also applies complex AI and machine
learning algorithms to scientific fields such as human health and
physics. Overall, the Summit is applying the power of super
computing software to solve problems in many scientific fields.

%%%%%%%%%%%%%%%%%%%%%%%%%%%%%%%%%%%%%%%%%%%%%%%%%%%%%%%%%%%%%%%%%%%%%%%%%%%%%%%%

\section{CONCLUSION}
At the beginning of the paper, we knew that supercomputers were simply computers that were much faster and much more powerful. However after going through and doing research, we have found that supercomputers are often used to advance science and to do the calculations that we as humans have difficulty doing with ease. While we expected to find that the supercomputers were used to advance research, we found out some interesting things that we didn't know before. One of the surprising things we found is that the first supercomputer had only 40 MHz of Clock Speed which is slower than the almost all modern laptops. It was also cooled with Freon, which is almost like water cooling. Another interesting thing we found was that the most recent supercomputer has over 27,000 GPUs which is quite impressive. Overall supercomputers are very useful to society and more specifically science itself, since it allows humans to do what we have done in years in mere seconds. This kind of technology is what is going to change the future for the better or the worst.

%%%%%%%%%%%%%%%%%%%%%%%%%%%%%%%%%%%%%%%%%%%%%%%%%%%%%%%%%%%%%%%%%%%%%%%%%%%%%%%%

\section*{REFERENCES}

\begin{enumerate}[label={[\arabic*]}]
\item Supercomputers. (n.d.). In computerhistory.org. Retrieved October 6, 2019, from https://www.computerhistory.org/revolution/supercomputers/10/76
\item CDC 6600. (n.d.). In cisl.ucar.edu. Retrieved October 6, 2019, from https://www2.cisl.ucar.edu/supercomputer/cdc6600
\item Staff. (2019, April 17). Vintage Video: The Paragon Supercomputer – A Product of Partnership. In insidehp.com. Retrieved October 6, 2019, from https://insidehpc.com/2019/04/vintage-video-the-paragon-supercomputer-a-product-of-partnership/
\item McCorkle, M. L. (2018, June 8). ORNL lauches Summit supercomputer. In ornl.gov. Retrieved October 6, 2019, from https://www.ornl.gov/news/ornl-launches-summit-supercomputer
\item Anthony, S. (2012, April 12). The history of supercomputers. Retrieved October 5, 2019, from https://www.extremetech.com/extreme/125271-the-history-of-supercomputers.
\item Intel XP/S 140 Paragon: Sandia National Labs. (n.d.). Retrieved October 5, 2019, from https://www.top500.org/featured/systems/intel-xps-140-paragon-sandia-national-labs/.
\item Summit. (n.d.). Retrieved October 5, 2019, from https://www.olcf.ornl.gov/olcf-resources/compute-systems/summit/.
\end{enumerate}

\end{document}
