%%%%%%%%%%%%%%%%%%%%%%%%%%%%%%%%%%%%%%%%%%%%%%%%%%%%%%%%%%%%%%%%%%%%%%%%%%%%%%%%
\documentclass[letterpaper, 10 pt, conference]{IEEEconf}

\title{\LARGE \bf
COMPUTER HISTORY\\
\large Supercomputers
}

\author{Group 13\\
\small Justin Markwell (Spikeums)\\
\small Jay Choate (BlueJay)\\
\small Owen Sortwell (PigeonStudent)\\
\small Wesley Camphouse (CoderCleric)\\
}

% Image/graphics support
\usepackage{graphicx}
\graphicspath{ {./images/} }

% Formatting for lists
\usepackage{enumitem}

% Formatting for media
\usepackage{float}
\restylefloat{table}
\restylefloat{figure}

\begin{document}


\maketitle
\thispagestyle{empty}
\pagestyle{empty}


%%%%%%%%%%%%%%%%%%%%%%%%%%%%%%%%%%%%%%%%%%%%%%%%%%%%%%%%%%%%%%%%%%%%%%%%%%%%%%%%
\section{Introduction}

Our group chose Supercomputers as our topic. We chose this simply because we thought it was interesting and we wanted to know more about them. A supercomputer is a computer that is much faster and more powerful than the average computers of the time period. They were usually used to do research and to run calculations that people could not do easily themselves. Often times supercomputers functioned as nodes, or groups, of computers. Another difference between normal and supercomputers is that supercomputers measure their computing power in flops such as gigaflops, petaflops, and so on rather than bytes like megabytes and gigabytes. Supercomputers also take in a massive amount of power. We decided to go with an approach to the project where we divide the topics under each of the three supercomputers we chose. We chose the CDC 6600 from 1964, the Intel Paragon from 1993, and the IBM Summit from 2019. 

%%%%%%%%%%%%%%%%%%%%%%%%%%%%%%%%%%%%%%%%%%%%%%%%%%%%%%%%%%%%%%%%%%%%%%%%%%%%%%%%
\section{CDC 6600 - 1964}
\subsection{Time Period}
\subsection{Hardware}
\subsection{Software}

%%%%%%%%%%%%%%%%%%%%%%%%%%%%%%%%%%%%%%%%%%%%%%%%%%%%%%%%%%%%%%%%%%%%%%%%%%%%%%%%
\section{Intel Paragon - 1993}
\subsection{Time Period}
\subsection{Hardware}
\subsection{Software}

%%%%%%%%%%%%%%%%%%%%%%%%%%%%%%%%%%%%%%%%%%%%%%%%%%%%%%%%%%%%%%%%%%%%%%%%%%%%%%%%
\section{Summit (IBM) - 2019}
\subsection{Time Period}
\subsection{Hardware}
\subsection{Software}

%%%%%%%%%%%%%%%%%%%%%%%%%%%%%%%%%%%%%%%%%%%%%%%%%%%%%%%%%%%%%%%%%%%%%%%%%%%%%%%%
\section{CONCLUSION}

At the beginning of the paper, we knew that supercomputers were simply computers that were much faster and much more powerful. However after going through and doing research, we have found that supercomputers are often used to advance science and to do the calculations that we as humans have difficulty doing with ease. While we expected to find that the supercomputers were used to advance research, we found out some interesting things that we didn't know before. One of the surprising things we found is that the first supercomputer had only 40 MHz of Clock Speed which is slower than the almost all modern laptops. It was also cooled with Freon, which is almost like water cooling. Another interesting thing we found was that the most recent supercomputer has over 27,000 GPUs which is quite impressive. Overall supercomputers are very useful to society and more specifically science itself, since it allows humans to do what we have done in years in mere seconds. This kind of technology is what is going to change the future for the better or the worst.

%%%%%%%%%%%%%%%%%%%%%%%%%%%%%%%%%%%%%%%%%%%%%%%%%%%%%%%%%%%%%%%%%%%%%%%%%%%%%%%%
\section*{REFERENCES}

\begin{enumerate}[label={[\arabic*]}]
\item Name of Author, (Year, Month),
Title of Internet Article [Online]. Available E-Mail:
E-Mail.

\end{enumerate}

\end{document}
