%%%%%%%%%%%%%%%%%%%%%%%%%%%%%%%%%%%%%%%%%%%%%%%%%%%%%%%%%%%%%%%%%%%%%%%%%%%%%%%%

% IEEEconf.cls file must exist in the same directory as the TeX file you want to compile
\documentclass[letterpaper, 10 pt, conference]{IEEEconf}

\title{\LARGE \bf
COMPUTER HISTORY\\
\large Group Topic Expressed In A Few Words
}

\author{Group 13\\
\small Justin Markwell\\
\small Jay Choate\\
\small Owen Sortwell\\
\small Wesley Camphouse\\
}

% Image/graphics support
\usepackage{graphicx}
\graphicspath{ {./images/} }

% Formatting for lists
\usepackage{enumitem}

% Formatting for media
\usepackage{float}
\restylefloat{table}
\restylefloat{figure}

\begin{document}


\maketitle
\thispagestyle{empty}
\pagestyle{empty}


%%%%%%%%%%%%%%%%%%%%%%%%%%%%%%%%%%%%%%%%%%%%%%%%%%%%%%%%%%%%%%%%%%%%%%%%%%%%%%%%
\section{Introduction}

This document is an example of Assignment 3 in CSE/IT101.
You should describe what topic you chose and
why you chose that topic. Also provide a summary of the
topic, in a general sense. This section should describe any
background information that the reader needs to know to
understand your topic.

%%%%%%%%%%%%%%%%%%%%%%%%%%%%%%%%%%%%%%%%%%%%%%%%%%%%%%%%%%%%%%%%%%%%%%%%%%%%%%%%
\section{CDC 6600 - 1964}
\subsection{Time Period}
\subsection{Hardware}
\subsection{Software}
NCAR's Computing Facility first acquired the CDC 6600 in 1964. 
Only a year later in 1965, the computing facility was behind in 
development for software for the CDC 6600. The two operating 
system available for use on the CDC were SPIROS (Simultaneous
Processing Operating System) and COS (Chippewa Operating System). 
Both operating systems were inefficient at running the code 
developed by NCAR. Therefore, they determined the computer would 
need it's own OS, and one was built using a 6600 simulator for 
another computer, the CDC 3600. In the end, before it's 
decommission, the CDC 6600 had it's own OS, program library, and 
FORTRAN 66 compiler. It also ran some early video games made by 
the NCAR engineers, including games similar to Lunar Lander, Space
Wars, and baseball.
%%%%%%%%%%%%%%%%%%%%%%%%%%%%%%%%%%%%%%%%%%%%%%%%%%%%%%%%%%%%%%%%%%%%%%%%%%%%%%%%
\section{Intel Paragon - 1993}
\subsection{Time Period}
\subsection{Hardware}
\subsection{Software}
The Intel Paragon was supplied by Intel to Sandia National Labs in
1994, with an OS designed by Intel. The OS provided by Intel,
OSF-1, failed to work well with the scale of the project at Sandia
National Labs. Instead of using OSF-1, the engineers at Sandia
National Labs ported their own OS over to the Intel Paragon,
SUNMOS. The kernel was designed specifically to be lightweight,
and it was later replaced by another lightweight kernel, PUMA.
Lightweight kernels were used, presumably, to save processing
power, so that the power of the supercomputer could be maximized.


\ref{tbl:example} for an example of a table.
The labels/captions of the table should be put at the bottom
of the table.


\begin{table}[h!]
\begin{center}
\begin{tabular}{||c | c | c | c||} 
\hline
  & Col1 & Col2 & Col3 \\ [0.5ex]
\hline\hline
Row1 & (1,1) & (1,2) & (1,3) \\ 
\hline
Row2 & (2,1) & (2,2) & (2,3) \\
\hline
Row3 & (3,1) & (3,2) & (3,3) \\
\hline
\end{tabular}
\caption{Example of a table}
\label{tbl:example}
\end{center}
\end{table}

You might want to put figures in the document. Please
remember to label them. The labels/captions of the figures
should be put at the bottom of the figure. See Figure
\ref{fig:example} for an example of how to use figures.
You will need to place the figure in an \texttt{images/} folder
in your working directory.

\begin{figure}[h!]
\centering
\includegraphics[width=0.5\textwidth]{spiral.png}
\caption{Example of a figure}
\label{fig:example}
\end{figure} 

\section{Summit (IBM) - 2019}
\subsection{Time Period}
\subsection{Hardware}
\subsection{Software}
For security reasons, Oak Ridge National Laboratories newest
supercomputer runs on an unknown and unreleased OS. The specific
software being used on the Summit is also unknown, however the
areas being affected by the computer are know. The computer's main
purpose is to run complex modeling programs in various scientific
fields, including biology, materials, and of course, computer
science. The Summit is the first computer to complete an exascale
(billion billion calculations per second) scientific calculation
in genomics. The Summit also applies complex AI and machine
learning algorithms to scientific fields such as human health and
physics. Overall, the Summit is applying the power of super
computing software to solve problems in many scientific fields.

Describe the software used for your chosen topic if any,
and state any uses of the software that your topic had.
If your topic does not have or use software, describe why it
doesn't use software and how it functions without it.

\section{CONCLUSION}

Conclude your research paper with any reflections on what you
learned about your topic. Was this what you expected to find?
Did you find any facts that surprised you? You may add other
personal reflections about the topic here.

\section*{REFERENCES}

Below are basic formats for different types of references.

\begin{enumerate}[label={[\arabic*]}]
\item CDC 6600. (n.d.). Retrieved from
https://www2.cisl.ucar.edu/supercomputer/cdc6600.
\item Intel XP/S 140 Paragon: Sandia National Labs. (n.d.).
Retrieved from https://www.top500.org/featured/systems/intel-xps-
140-paragon-sandia-national-labs/.
\item ORNL Launches Summit Supercomputer. (2018, June 8).
Retrieved from 
https://www.ornl.gov/news/ornl-launches-summit-supercomputer.
\end{enumerate}

\end{document}
